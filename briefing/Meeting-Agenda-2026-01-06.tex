\documentclass[11pt]{article}

\usepackage[margin=1in]{geometry}
\usepackage{booktabs}
\usepackage{hyperref}
\usepackage{lmodern}
\usepackage{xcolor}

% Time formatting for timestamp
\newcount\myhours
\newcount\myminutes
\myhours=\time \divide\myhours by 60
\myminutes=\time \multiply\myhours by 60 \advance\myminutes by -\myhours
\divide\myhours by 60
\def\mytime{\ifnum\myhours<10 0\fi\the\myhours:\ifnum\myminutes<10 0\fi\the\myminutes}

% Colors
\definecolor{sectionblue}{RGB}{0,82,147}

\title{\vspace{-2em}\textbf{SpeakUp Project}\\[0.3em]\Large Customer Briefing Meeting Agenda}
\author{Bruce Dombrowski}
\date{January 6, 2026}

\begin{document}

\maketitle

\begin{center}
\fbox{\parbox{0.9\textwidth}{
\centering
\textbf{Repository:} \url{https://github.com/brucedombrowski/SpeakUp}\\[0.5em]
\textit{All artifacts, source code, and verification evidence available for review}
}}
\end{center}

\vspace{0.5em}
\begin{center}
\footnotesize Generated: \today\ \mytime\ UTC
\end{center}

\vspace{1em}

%======================================================================
\section*{1. Meeting Overview}
%======================================================================

\begin{tabular}{@{}ll@{}}
\textbf{Duration:} & 45--60 minutes \\
\textbf{Format:} & Presentation + Live Demo \\
\textbf{Audience:} & Technical leadership, program managers, customers \\
\end{tabular}

\vspace{0.5em}

\textbf{Meeting Objectives:}
\begin{itemize}
    \item Demonstrate a systems-engineering approach to knowledge work
    \item Show working Bundle Protocol (BPv7) implementation with live network capture
    \item Present workflow model for improved productivity and traceability
    \item Provide verification evidence and compliance documentation
\end{itemize}

%======================================================================
\section*{2. Agenda}
%======================================================================

\begin{table}[h]
\centering
\begin{tabular}{@{}clp{8cm}@{}}
\toprule
\textbf{Time} & \textbf{Topic} & \textbf{Description} \\
\midrule
5 min & Introduction & Project genesis and governing principles \\
10 min & Problem Statement & Eight systemic constraints limiting effectiveness \\
10 min & Proposed Solution & Workflow model: Ideation $\rightarrow$ Execution $\rightarrow$ System of Record \\
15 min & \textbf{Live Demo} & Wireshark capture of BPv7 bundle transmission \\
5 min & Verification & Security scans, compliance evidence, traceability \\
5 min & Value Proposition & One person, entire team's output \\
5 min & Next Steps \& Q\&A & Discussion and path forward \\
\bottomrule
\end{tabular}
\end{table}

%======================================================================
\section*{3. Key Files to Share}
%======================================================================

\textbf{Before the meeting}, have these ready:

\begin{table}[h]
\centering
\begin{tabular}{@{}lp{9cm}@{}}
\toprule
\textbf{File} & \textbf{Purpose} \\
\midrule
\texttt{README.md} & Authoritative project specification and requirements \\
\texttt{briefing/SpeakUp-Briefing.pdf} & Executive briefing deck (main presentation) \\
\texttt{verification/Compliance-Statement.md} & Information handling verification \\
\texttt{verification/Requirements-Traceability.md} & Requirements-to-evidence matrix \\
\bottomrule
\end{tabular}
\end{table}

\textbf{For the live demo}, have these ready:

\begin{table}[h]
\centering
\begin{tabular}{@{}lp{9cm}@{}}
\toprule
\textbf{File} & \textbf{Purpose} \\
\midrule
\texttt{src/bpv7/capture\_demo.sh} & Wireshark capture demo script \\
\texttt{src/bpv7/test\_local.py} & Console-based bundle test with hex dump \\
\texttt{src/bpv7/test\_pdf\_transfer\_tshark.py} & PDF transfer with tshark verification \\
\bottomrule
\end{tabular}
\end{table}

%======================================================================
\section*{4. Presentation Script}
%======================================================================

%----------------------------------------------------------------------
\subsection*{4.1 Introduction (5 minutes)}
%----------------------------------------------------------------------

\textit{Key talking points:}

\begin{quote}
``SpeakUp is a response to two calls---our organization's call for employees to speak up constructively, and our customer's request for process improvement recommendations.

This project demonstrates that \textbf{with the right environment, one person can do the work of an entire team}. Everything you'll see today---the briefing, the code, the verification evidence---was produced by one person using the workflow model we're proposing.

The project is:
\begin{itemize}
    \item \textbf{Concrete enough to execute}---working code, real artifacts
    \item \textbf{Abstract enough to remain vendor-neutral}---no specific tools required
    \item \textbf{Self-demonstrating}---created using the workflow it proposes
\end{itemize}
''
\end{quote}

%----------------------------------------------------------------------
\subsection*{4.2 Problem Statement (10 minutes)}
%----------------------------------------------------------------------

\textit{Walk through the eight constraints} (Slide: Problem Statement):

\begin{enumerate}
    \item Fragmented workflows across mobile, desktop, and execution environments
    \item AI assistance unavailable or outside trust boundaries
    \item Inbox-centric work with decisions buried in email
    \item Untracked coordination limiting traceability
    \item Knowledge attrition as personnel retire
    \item Budget constraints reducing investment
    \item Legacy systems facing decommissioning
    \item Regulatory burden diverting resources
\end{enumerate}

\textit{Emphasize:} These are \textbf{systemic} constraints---not individual failures. The solution requires environmental change, not harder work.

%----------------------------------------------------------------------
\subsection*{4.3 Proposed Solution (10 minutes)}
%----------------------------------------------------------------------

\textit{Walk through the workflow model} (Slide: Proposed Workflow Model):

\begin{quote}
``The workflow has three phases:

\textbf{1. Ideation (Mobile):} Frame objectives, capture rough drafts, use AI to synthesize. This happens anywhere---on a phone, tablet, desktop.

\textbf{2. Execution (IDE):} Turn drafts into versioned, reviewable artifacts. Build inside a development environment with AI assistance. Execute within trusted boundaries.

\textbf{3. System of Record (Git):} Everything is captured automatically. History, rationale, decisions---all preserved for audit and knowledge transfer.

The key insight is that \textbf{work flows freely between ideation and execution}, but \textbf{recording happens automatically}. No extra effort required.''
\end{quote}

%----------------------------------------------------------------------
\subsection*{4.4 Live Wireshark Demo (15 minutes)}
%----------------------------------------------------------------------

\textit{This is the technical highlight.} You'll demonstrate Bundle Protocol v7 actually on the wire.

\vspace{0.5em}
\textbf{Demo Setup:}
\begin{enumerate}
    \item Open Terminal
    \item Navigate to repository: \texttt{cd \textasciitilde/SpeakUp}
    \item Have Wireshark installed (or use tshark)
\end{enumerate}

\vspace{0.5em}
\textbf{Option A: Full Wireshark Demo (Recommended)}

\begin{verbatim}
# Run the capture demo
./src/bpv7/capture_demo.sh
\end{verbatim}

\textit{Explain while it runs:}
\begin{quote}
``What you're seeing is:
\begin{enumerate}
    \item tcpdump capturing traffic on the loopback interface, port 4556
    \item A BPv7 bundle being created and sent via TCPCL
    \item The capture file opening in Wireshark
\end{enumerate}

In Wireshark, I'll decode TCP port 4556 as TCPCL. You can see:
\begin{itemize}
    \item The `dtn!' magic bytes in the contact header
    \item CBOR-encoded bundle data
    \item The payload traversing a simulated DTN link
\end{itemize}
''
\end{quote}

\textbf{Option B: Console Demo (No Wireshark GUI)}

\begin{verbatim}
# Run console-based test with hex dump
PYTHONPATH=src python3 src/bpv7/test_local.py
\end{verbatim}

\textit{This shows:}
\begin{itemize}
    \item Bundle creation with CBOR encoding
    \item Hex dump of the bundle on the wire
    \item TCPCL contact header exchange
    \item Bundle reception and decoding
\end{itemize}

\textbf{Option C: PDF Transfer with tshark Verification}

\begin{verbatim}
# Transfer the briefing PDF via bundle protocol
PYTHONPATH=src python3 src/bpv7/test_pdf_transfer_tshark.py
\end{verbatim}

\textit{Key points to highlight:}
\begin{quote}
``This test sends the actual SpeakUp briefing PDF via Bundle Protocol. We're using tshark to capture and verify:
\begin{itemize}
    \item TCPCL contact headers are on the wire (`dtn!' magic)
    \item CBOR-encoded bundle payload is visible
    \item MD5 integrity is verified end-to-end
    \item Standards compliance: RFC 9171 (BPv7), RFC 9174 (TCPCL), RFC 8949 (CBOR)
\end{itemize}

This proves the implementation actually works---not just unit tests, but \textbf{observable protocol behavior on the network}.''
\end{quote}

\vspace{0.5em}
\textbf{Wireshark Deep Dive (if time permits):}
\begin{enumerate}
    \item Open the saved .pcap file
    \item Right-click TCP stream $\rightarrow$ Decode As $\rightarrow$ TCPCL
    \item Filter: \texttt{tcpcl} or \texttt{bpv7}
    \item Show the protocol hierarchy
    \item Point out the `dtn!' magic (hex: 64 74 6e 21)
    \item Show CBOR array start (0x9F)
\end{enumerate}

%----------------------------------------------------------------------
\subsection*{4.5 Verification Evidence (5 minutes)}
%----------------------------------------------------------------------

\textit{Quick walkthrough of verification artifacts:}

\begin{quote}
``The project produces verification evidence as first-class artifacts. This isn't afterthought documentation---it's built into the workflow.

\textbf{Automated Scans:}
\begin{itemize}
    \item PII scan---no sensitive personal data
    \item Malware scan---ClamAV clean
    \item Secrets scan---no API keys or credentials
    \item MAC address scan---no hardware identifiers
\end{itemize}

\textbf{Compliance:}
\begin{itemize}
    \item No CUI, no proprietary data, no classified information
    \item NIST SP 800-53 control alignment documented
    \item FIPS 199 security categorization: LOW
\end{itemize}

The key point: \textbf{you can review this repository without relaxing any security rules}.''
\end{quote}

%----------------------------------------------------------------------
\subsection*{4.6 Value Proposition (5 minutes)}
%----------------------------------------------------------------------

\textit{Emphasize the core message:}

\begin{quote}
``With the right environment, one person can do the work of an entire team.

What you've seen today:
\begin{itemize}
    \item A complete BPv7 protocol implementation
    \item TCP Convergence Layer for DTN communications
    \item Professional briefing deck (LaTeX to PDF)
    \item Verification and compliance evidence
    \item Video production pipeline
    \item Auditable cost tracking
\end{itemize}

All produced by \textbf{one person in approximately 8 hours}.

The constraint isn't capability. It's environment. If we provide the right tools, workflows, and trust boundaries, we can dramatically multiply productivity while maintaining---or improving---security and traceability.''
\end{quote}

%----------------------------------------------------------------------
\subsection*{4.7 Next Steps \& Q\&A (5 minutes)}
%----------------------------------------------------------------------

\textit{Proposed next steps:}

\begin{enumerate}
    \item \textbf{Review the repository}---all source, documentation, and evidence is available
    \item \textbf{Identify a pilot area}---where could this workflow be applied?
    \item \textbf{Establish infrastructure}---repository, IDE access, AI tooling within boundaries
    \item \textbf{Measure and iterate}---track outcomes, refine approach
\end{enumerate}

\textit{Open for questions.}

%======================================================================
\section*{5. Demo Commands Quick Reference}
%======================================================================

\begin{table}[h]
\centering
\begin{tabular}{@{}lp{9cm}@{}}
\toprule
\textbf{Command} & \textbf{Description} \\
\midrule
\texttt{./src/bpv7/capture\_demo.sh} & Full Wireshark capture demo \\
\texttt{PYTHONPATH=src python3 src/bpv7/test\_local.py} & Console bundle test with hex dump \\
\texttt{PYTHONPATH=src python3 src/bpv7/test\_pdf\_transfer\_tshark.py} & PDF transfer with tshark \\
\texttt{./build.sh} & Run full build (tests, scans, briefing, video) \\
\texttt{./verification/scripts/run-all-scans.sh} & Run security verification scans \\
\bottomrule
\end{tabular}
\end{table}

%======================================================================
\section*{6. Backup Materials}
%======================================================================

\textbf{If questions arise about:}

\begin{itemize}
    \item \textbf{Standards:} Reference RFC 9171 (BPv7), RFC 9174 (TCPCL), RFC 8949 (CBOR)
    \item \textbf{Security:} Show \texttt{verification/Security-Attestation.md}
    \item \textbf{Costs:} Run \texttt{./accounting/calculate-costs.sh}
    \item \textbf{Architecture:} Walk through \texttt{src/bpv7/} directory structure
    \item \textbf{Requirements:} Show \texttt{verification/Requirements-Traceability.md}
\end{itemize}

\vspace{2em}

\begin{center}
\rule{0.5\textwidth}{0.5pt}\\[1em]
\textit{This agenda was created using the SpeakUp workflow model it describes.}\\[0.5em]
\textbf{Repository:} \url{https://github.com/brucedombrowski/SpeakUp}
\end{center}

\end{document}
